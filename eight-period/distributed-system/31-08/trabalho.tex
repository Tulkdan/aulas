\documentclass[a4paper, 12pt]{article}

\usepackage{fontspec}
\defaultfontfeatures{Ligatures=TeX}
\usepackage{indentfirst}
\usepackage[brazilian]{babel}

\usepackage{textcomp}

%enconding
\usepackage[utf8]{inputenc}
\usepackage[T1]{fontenc}

\author{
  Pedro Correa\\
  \texttt{11718563} - \texttt{pedro.figueiredo563@al.faj.br}
}

\title{Biblioteca de sistema distrubuído}

\begin{document}

\maketitle

\newpage

Decidi escolher não uma biblioteca mas um método de comunicação que esta auxilia bastante em trabalhar com sistemas distribuídos pois eles facilitam o trabalho do desenvolvedor ao `padroniza' a comunicação entre dois sistemas.
Como os sistemas distribuídos estão em máquinas diferentes, é necessário um método de comunicação em comum para que eles consigam trocar informações,
muitos utilizam o HTTP, porém há um trabalho imenso entre os desenvolvedores sobre os \emph{payloads} que devem ser enviados, códigos de erros e o que cada um significa,
conforme o crescimento do sistema e da empresa, tratar de tudo isso se torna um inferno.
Além de que cada sistema pode ser escrito em uma linguagem diferente por conta de sua proposta.

\section{O que é gRPC?}

gRPC é um método de comunicação que foi desenvolvido pela Google para facilitar e deixar eficiente a comunção entre aplicações de diferentes linguagens e distribuídas se comunicarem mais rapidamente.

Essa comunicação utiliza o RPC, que significa \emph{Remote Procedure Calls} (Chamada de Procedimento Remotas em tradução livre).
Este protocolo é a uma das peça central do por que esse método de comunicação venho para facilitar a vida dos desenvolvedores,
ele basicamente deixa o desenvolvedor chamar um procedimento de outro programa que esteja em outra máquina,
com isso o desenvolvedor não precisa deixar explícito a requisição, a própria biblioteca se encarrega de fazer a chamada.

Mas como essa parte é realizada? Como o gRPC sabe qual serviço deve ser chamado para executar o procedimento?
Cada serviço deverá possuir um \emph{stub} que irá providenciar os mesmos métodos do servidor que devem ser expostos para os outros serviços.
Com isso, a instância do gRPC do seu serviço saberá qual serviço deverá ser chamado,
toda essa comunicação e mapeamento é gerenciado pela própria biblioteca.

\section{HTTP/2}

Outra peça central do gRPC é o HTTP/2, este protocolo é a versão mais otimizada do protocolo que tanto utilizamos, o HTTP, que esta na versão HTTP 1.1, porém popularmente chamamos de apenas HTTP.

E por que o gRPC utiliza ele ao invés do puro HTTP?
Bom, um dos fatores que venho para a criação do gRPC é que a comunicação seja extremamente rápida,
e como o HTTP transfere texto através da rede, para comunicação de muita informação acaba sendo muito lenta essa comunicação.

Com isso em mente, foi-se criado o HTTP/2 para que ao invés de ser transmitido texto pela rede, seja transmitido bits,
assim a comunicação de muita informação deverá ser mais rápida pois estamos trabalhando com a camada mais baixa da rede.

\section{Requisitos não funcionais}

A seguir segue a lista dos requisitos não funcionais aplicados pela biblioteca gRPC.

\begin{itemize}
  \item \textbf{Desempenho}:
    Como comentado anteriormente, a ideia da criação do gRPC é ser uma comunicação seja rápida por se tratar de muita informação que é necessária ser transmitida, ele se beneficia bastante do HTTP/2 para este requisito.
  \item \textbf{Eficiência}:
    Como comentado anteriormente, outra ideia do gRPC é ter uma comunicação eficiente, sem que os desenvolvedores precisem se preocupar em saber como deve ser feito a chamada para o serviço, podem simplesmente fazer a chamada como se fosse uma chamada de um método na linguagem de programação que ele preferir.
  \item \textbf{Manutenabilidade}:
    Como o responsável por expor e cuidar dos endpoints é a própria biblioteca através dos \emph{stubs}, acaba facilitando ainda mais a manitenção e versionamento de um endpoint, pois nem o desenvolvedor que irá utilizar ou a pessoa que irá fazer a manutenção, para acrescentar ou retirar um campo, precisam saber sobre como esta feita a implementação.
  \item \textbf{Portabilidade}:
    Atualmente esta bilbioteca esta disponível, com suporte oficial, para 11 linguagens de programação (Android Java, C\#/.NET, C++, Dart, Go, Java, Kotlin, Node.js, Objective-C, PHP, Python e Ruby), então independente do sistema operacional ou da capacidade técnica de seu time, a biblioteca poderá ser utilizada por eles.
  \item \textbf{Reusabilidade}:
    Possui facilidade para ser utilizada e reutilizada em várias sistemas, sem precisar deixar explícito o canal de comunicação.
\end{itemize}

\end{document}

