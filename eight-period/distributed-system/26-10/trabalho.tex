\documentclass[a4paper, 12pt]{article}

\usepackage{url}

\usepackage{fontspec}
\defaultfontfeatures{Ligatures=TeX}
\usepackage{indentfirst}
\usepackage[brazilian]{babel}

\usepackage{textcomp}

%enconding
\usepackage[utf8]{inputenc}
\usepackage[T1]{fontenc}

\author{
  Pedro Correa\\
  \texttt{11718563} - \texttt{pedro.figueiredo563@al.faj.br}
}

\title{Saas ou On-Premises}

\begin{document}

\maketitle

\newpage

\section{Software as a Service (SaaS)}

Nos anos 1990, com a popularidade e expansão da internet, começou a surgir a ideia de centralizar toda a computação, essa ideia ficou conhecida como \emph{application service providers} (ASP).
A ideia dessas ASPs seria prover serviços de hosting e gerenciamento de aplicações especializados com o objetivo de reduzir os custos através de um central de administração.

O \emph{Software as a Service} (SaaS) extende essa definição do ASP. Porém ele é mais utilizado nas seguintes configurações:

\begin{itemize}
  \item Enquanto a maioria dos ASPs focava no gerenciamento e armazenar aplicações de terceiros, no SaaS geralmente são os contratados que desenvolvem e gerenciam o seu próprio \emph{software};
  \item Enquanto a maioria dos ASPs oferecia aplicações client-server, as soluções utilizando o SaaS são mais predominante na web e apenas requer um web browser para navegar;
\end{itemize}

Como a forma de distribuição do SaaS é de forma indireta e não é fisicamente,
a forma de pagamento para se utilizar este serviço acaba sendo totalmente diferente do software que necessitam uma licença para ser utilizado e instalado.
Com isto muitos provedores optaram pelo método de cobrar utilizando um assinatura mensal para que o contratante consiga se beneficiar de todos as \emph{features} disponíveis pela empresa.
Desta forma eles conseguem fazer assinaturas por um preço muito mais baixo que uma licença de um programa.

Também existe a possibilidade de cobrar o cliente por eventos,
transações e outras unidades que possam gerar valor,
um exemplo seria a quantatidade de processadores que estão sendo utilizados ou por tempo que um serviço utiliza de computação.

A grande maioria das soluções que utilizam o SaaS utilizam a arquitetura baseada no \emph{multitenant}.
O modelo simplesmente é para quando se utiliza uma única versão da aplicação e que serve vários clientes.
Esta arquitetura facilita muito na escalabilidade do produto para ser instalado em múltiplas máquinas,
e também facilita muito quando se deve testar novas funcionalidades que deve ser somente para um grupo de usuários.

\section{On-Premises}

\emph{Softwares} considerados On-Premises são \emph{softwares} que são instalados dentro da empresa, ou seja,
são instalados em um \emph{hardware} que a empresa possui e que servirá como um servidor que irá distribuir o serviço que for instalado nele.

Com isto, a empresa deverá ter total controle sobre como poderá escalar e também deverá ter o cuidade de fazer manutenção do servidor contra ataques que possivelmente possam ocorrer.

\section{Diferença entre SaaS e On-Premises}

A principal diferença entre os dois é simplesmente o modo de onde tudo será armazenado e utilizado para servir.
Com o SaaS, como o serviço acaba sendo disponibilizado por um serviço de terceiro e tudo isso é protegido por um contrato e assinatura,
acaba se tornando uma solução bem barata e simples caso a empresa não queira arcar com os custos de sustentar um servidor em sua rede,
pois com isso deverá ter profissionais especializados em segurança, tratar de backups para não perder informações dos clientes e além de ter que possuir uma infra que suporte toda a demanda que seus clientes poderão ter.

Porém um dos pontos fracos do SaaS em relação ao On-Premises é que a customização que o usuário possui é muito baixa,
como os sistemas On-Premises são de propriedade da empresa por pagar uma licença por possuir o \emph{software}, ela consegue personalizar conforme a sua necessidade.
Outro ponto forte em relação ao On-Premises é em relação as informações dos clientes da empresa, como o SaaS é outra empresa que está disponibilizando os serviços,
eles podem ter acesso as informações da empresa que esteja traferrando entre os serviços.
Assim, cabe a empresa se não for um problema tão grande assim as informações dela serem vazadas ou outras pessoas de fora dela poderem acessar tais informações.

Para a escolha da empresa de qual utilizar ele deve levar em considerações os pontos citados acima, pois são esses fatores em especial que separa e torna cada uma das soluções especiais.
Nenhum é a bala de prata que irá resolver todas as situações, cabe a empresa se quer ter todo o trabalho de dar suporte para um servidor e ter mais controle sobre as informações,
então vale muito mais a pena ela escolher a solução do On-Premises.
Caso ela não queira arcar com essas dívidas que podem, do dia pra noite se tornar uma super dor de cabeça para a empresa, seria mais favorárel escolher o SaaS como solução.

\nocite{*}
\bibliography{ref}
\bibliographystyle{plain}

\end{document}

