\section{Análise Semântica} 

Esta etapa é responsável por verificar aspectos relacionados ao significado das instruções e também a validação sobre uma série de regras que não podem ser verificadas nas etapas anteriores.

As validações que não são realizadas durante as etapas anteriores são realizadas durante a análise semântica a fim de garantir que o programa fonte esteja coerente e o mesmo possa ser convertido para a linguagem de máquina.
A análise semântica percorre a árvore sintática e relaciona os identificadores com sues dependentes de acordo com a estrutura hierárquica.

Esta etapa também é responsável pela captura de informações sobre o programa fonte para que as fases subsequentes gerem o código objeto,
outro fator importante desta fase é a validação de tipos,
nela o compilador verifica se cada operador recebe os operandos permitidos e especificados na linguagem fonte.

Os tipos de dados são muito importantes nessa etapa da compilação, pois eles são os responsáveis por dar características para as linguagens de programação.
Com base nos tipos, o analisado semântico pode definir quais valores podem ser manipulados, essa ação é conhecida como \emph{type checking}.

\subsection{Inferência de tipos}

O sistema de tipos de dados podem ser divididos em dois grupos: sistemas dinâmicos e sistemas estáticos.
Muitas das linguagens utilizam o sistema estático pois essa informação é utilizada durante a compilação e simplifica o trabalho do compilador.
Esse sistema é muito predominante em linguagens compiladas.

\begin{itemize}
  \item \textbf{Sistema de tipo estático}: obrigam o programador definir os tipos das variáveis e retorno de funções, exemplos de linguagens: C, Java, Pascal.
  \item \textbf{Sistema de tipo dinâmico}: variáveis e retorno de funções não possuem declaração de tipos, exemplos de linguagens: Python, PHP, Javascript.
\end{itemize}

Algumas linguagens utilizam um mecanismo muito interessante chamado inferẽncia de tipos,
que permite a uma variável assumir vários tipos durante o seu ciclo de vida, permitindo que ela consiga ter vários valores.
Nesses casos o compilador infere o tipo da variável em tempo de execução, esse mecanismo esta diretamente relacionado ao mecanismo de \emph{Generics} do Java.
Linguagens de programação como o Haskel utilizam esse método.
